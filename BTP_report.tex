%=====================================================
% LastMile Service BTP Report - MAIN FILE
%=====================================================
\documentclass[12pt,a4paper]{report}

%------------ Packages -------------
\usepackage[utf8]{inputenc}
\usepackage[T1]{fontenc}
\usepackage{setspace}
\usepackage{lmodern}
\usepackage{graphicx}
\usepackage{amsmath, amssymb}
\usepackage{booktabs}
\usepackage{enumitem}
\usepackage{hyperref}
\usepackage{geometry}
\usepackage{float}
\usepackage{caption}
\usepackage{subcaption}
\usepackage{array}
\usepackage{xcolor}

% All BTP visuals are stored here on this machine
\graphicspath{{C:/Users/sunil/Desktop/BTP_visuals/}}

\geometry{
  left=1in,
  right=1in,
  top=1in,
  bottom=1in
}

\onehalfspacing

\hypersetup{
    colorlinks=true,
    linkcolor=black,
    urlcolor=blue,
    citecolor=black
}

%------------ Metadata -------------
\title{Design and Development of a Trust-Based Digital Marketplace\\
for Brand-Certified Appliance Services in Rural India}

\author{Sunil (2022516) \and Sumit (2022515) \and Priyanshu (2022382) \and Devesh (2022155)\\[4pt]
B.Tech in Computer Science and Engineering\\
Indraprastha Institute of Information Technology Delhi (IIIT-Delhi)}

\date{April 2026}

%=====================================================
\begin{document}

% If your institute gives a cover file, include it here:
% \input{btech_thesis_cover}

\maketitle
\pagenumbering{roman}

%-----------------------------------------------------
\chapter*{Students' Declaration}
\addcontentsline{toc}{chapter}{Students' Declaration}

We hereby declare that the work presented in the report entitled
\textbf{``Design and Development of a Trust-Based Digital Marketplace for Brand-Certified Appliance Services in Rural India''},
submitted by us in partial fulfillment of the requirements for the degree of Bachelor of Technology in Computer Science and Engineering at
Indraprastha Institute of Information Technology Delhi (IIIT-Delhi), as part of the Entrepreneurship B.Tech Project (BTP),
is an authentic record of our work carried out under the guidance of
\textbf{Prof.\ Pankaj Vajpayee}. All sources of information used in this work have been duly acknowledged in the report.
This work has not been submitted elsewhere for the award of any degree or diploma.

\vspace{1.5cm}
\noindent
\textbf{Place:} New Delhi\\
\textbf{Date:} 25 November 2025 \\[1cm]

\begin{flushright}
\textbf{Sunil (2022516)}\\
\textbf{Sumit (2022515)}\\
\textbf{Priyanshu (2022382)}\\
\textbf{Devesh (2022155)}\\
\end{flushright}

%-----------------------------------------------------
\chapter*{Certificate}
\addcontentsline{toc}{chapter}{Certificate}

This is to certify that the above statement made by the candidates is correct to the best of my knowledge and that this report,
entitled \textbf{``Design and Development of a Trust-Based Digital Marketplace for Brand-Certified Appliance Services in Rural India''},
is a bona fide record of the work carried out by them under my supervision and guidance as part of the Entrepreneurship B.Tech Project at IIIT-Delhi.

\vspace{1.5cm}
\noindent
\textbf{Place:} New Delhi \hfill \textbf{Prof.\ Pankaj Vajpayee}\\
\textbf{Date:} 25 November 2025\hfill Faculty, Entrepreneurship, IIIT-Delhi

%-----------------------------------------------------
\chapter*{Abstract}
\addcontentsline{toc}{chapter}{Abstract}

In the past decade, the penetration of electrical and electronic appliances in rural and semi-urban India has grown rapidly.
Fans, mixers, televisions, refrigerators and smartphones have become embedded in daily routines, improving comfort, productivity and social participation.
However, the availability of reliable, affordable and timely after-sales service has not kept pace with this growth.
Households often depend on informal local technicians, must travel long distances to brand service centres, or simply discard repairable products.
This results in financial stress, unnecessary e-waste and a gradual erosion of trust in both brands and service providers.

This B.Tech project proposes and develops \textbf{LastMile Service} --- a trust-based digital marketplace that connects three key actors:
(1) rural and semi-urban consumers who need repair or maintenance services for appliances,
(2) local technicians who can deliver on-ground service, and
(3) appliance and device brands that wish to extend their after-sales reach in rural India.
The platform aims to combine the familiarity and accessibility of local repair with brand-aligned training, transparent pricing and a structured reputation system.

The work in this project spans three interlinked dimensions.
First, \emph{problem understanding and field research}: a mixed-method study using an online survey (58 respondents) and semi-structured interviews with consumers, technicians and retailers is used to map the current after-sales journey, identify pain points and understand how trust is formed or broken.
Second, \emph{system design and engineering}: we design and implement a full-stack web platform that supports booking, intelligent technician matching, job tracking and feedback, with explicit attention to rural constraints such as intermittent connectivity and low-end smartphones.
Third, \emph{business and impact analysis}: we explore partnership models with brands, outline possible revenue streams and unit economics, and simulate a small-scale pilot using realistic workflows and dummy users.

The implemented minimum viable product (MVP) supports customer and technician registration, creation of service requests, automated matching to nearby technicians based on distance, skills and reputation, real-time status tracking, and feedback collection through ratings and short reviews.
Event logs are used to compute operational metrics such as average response time, job completion rate and technician utilisation, which are relevant to both platform operations and brand reporting.

Overall, the project demonstrates that a structured, trust-based digital platform can make last-mile appliance service more accessible and reliable for rural households, while providing better livelihood opportunities for technicians and a scalable extension channel for brands.
It also surfaces practical limitations and future directions for taking such a platform from academic prototype to real-world deployment.

\begin{figure}[H]
    \centering
    \includegraphics[width=0.7\textwidth]{img2.jpg}
    \caption{High-level overview of the LastMile Service concept: connecting rural households, local technicians and brand partners through a digital platform.}
    \label{fig:lastmile_concept_overview}
\end{figure}

%-----------------------------------------------------
\chapter*{Acknowledgements}
\addcontentsline{toc}{chapter}{Acknowledgements}

We would like to express our sincere gratitude to our advisor, \textbf{Prof.\ Pankaj Vajpayee}, for his constant guidance, critical feedback and encouragement throughout this project.
His support helped shape the idea of LastMile Service from a vague frustration into a structured engineering and entrepreneurship problem.

We are thankful to the technicians, shop owners and rural households who generously shared their time and experiences in interviews and phone calls.
Their stories and constraints grounded the design of this platform in real-world needs rather than assumptions.

We are also grateful to our friends and batchmates for brainstorming sessions, late-night debugging help, and for acting as early users during testing.
Finally, we thank our families for their patience, motivation and unwavering support during the ups and downs of this project.

%-----------------------------------------------------
\tableofcontents
\listoffigures
\listoftables

\clearpage
\pagenumbering{arabic}


%=====================================================
\chapter{Introduction}
\label{chap:intro}

\section{Background}

Over the last few years, the availability of electrical and electronic appliances in rural and semi-urban India has increased significantly.
Fans, mixers, televisions, refrigerators and smartphones are now considered essential for comfort, productivity and social inclusion.
This growth has been driven by declining hardware costs, rising rural incomes, government programmes related to electrification and digital access, and aggressive distribution strategies by major brands.

However, after-sales service infrastructure has not expanded at the same pace in many districts.
Urban customers can easily book brand-authorised service via websites, apps or call centres, and often receive predictable appointment slots.
In contrast, a rural household may need to travel 30--50 kilometres to the nearest service centre, wait multiple days for spare parts, and deal with uncertain costs.
For frequent, smaller failures such as fan capacitor issues or mixer motor problems, the gap is often filled by local informal technicians.
These technicians are accessible and low-cost, but their skills, tools and access to genuine parts vary widely.

This mismatch creates a classic ``last-mile'' problem: products reach rural households, but reliable servicing does not.
The consequences go beyond inconvenience --- they induce financial strain when appliances fail, increase e-waste when repairable devices are discarded, and gradually erode trust in both brands and local technicians.

\section{Motivation and Origin of the Idea}

The motivation for this project is rooted in lived experiences and repeated anecdotes from relatives and contacts in small towns and villages.
Stories of ceiling fans lying unrepaired for months, refrigerators transported to district headquarters on rented vehicles, and smartphones replaced rather than repaired because of uncertainty and mistrust are common.
On the other side, conversations with local technicians reveal a different frustration: they often possess strong practical skills but remain invisible to brands, lack formal recognition, and depend heavily on a small circle of shopkeepers for work.

Existing service aggregators and platforms --- typically designed for metropolitan users --- offer a polished digital experience but make strong assumptions about reliable connectivity, digital payments and dense technician supply.
Their business and product choices rarely reflect rural constraints such as patchy networks, cash-based transactions or shared smartphones within a household.
As a result, rural consumers and technicians remain peripheral to mainstream digital service ecosystems.

These observations led to a simple but powerful question:
\emph{Can we design a platform that takes rural realities seriously, leverages existing local technicians, and still preserves brand-level trust and quality?}
LastMile Service is an attempt to explore this question through the lens of engineering, product thinking and entrepreneurship.

\section{Ideology and Vision Behind LastMile Service}

The ideology behind LastMile Service combines three threads:

\begin{itemize}
    \item \textbf{Trust as a first-class feature:} In many rural contexts, trust in a technician or shopkeeper is built over years of interaction.
    The platform is designed not to replace this trust but to make it visible, portable and accountable through ratings, verifications and transparent histories.
    \item \textbf{Respect for local expertise:} Rather than assuming that value must flow from a central brand to a passive village, LastMile treats local technicians as partners.
    The long-term vision is to support their upskilling, documentation and formal recognition.
    \item \textbf{Brand alignment without over-centralisation:} Brands care about reputation, warranty control and data.
    The platform aims to provide them with visibility and governance hooks without forcing them to build their own rural micro-infrastructure from scratch.
\end{itemize}

In the long run, the vision for LastMile Service is to become an infrastructure layer for rural appliance service --- a common protocol and marketplace through which multiple brands, technicians and communities can coordinate, rather than a single-brand captive app.

\section{Problem Statement}

The central problem addressed in this project can be stated as follows:

\medskip
\noindent
\textbf{Problem Statement:}\\
\emph{Rural and semi-urban households in India lack a reliable, convenient and trustworthy mechanism
to access brand-quality repair and maintenance services for household appliances,
while local technicians and appliance brands lack a structured channel to discover each other,
coordinate jobs and build reputations.}

\section{Research Questions}

To explore this problem space in a disciplined way, the project is guided by the following research questions:

\begin{itemize}
    \item \textbf{RQ1:} How do rural and semi-urban consumers currently discover technicians and make decisions about repair versus replacement?
    \item \textbf{RQ2:} What are the main trust gaps between consumers, local technicians and brand service networks?
    \item \textbf{RQ3:} What minimum set of digital workflows (booking, matching, tracking, feedback) can realistically be adopted in rural contexts?
    \item \textbf{RQ4:} How can we design a matching and reputation system that respects local constraints (distance, connectivity, skills) while remaining simple to understand?
    \item \textbf{RQ5:} Under what conditions could such a platform be sustainable for technicians, attractive to brands and affordable for consumers?
\end{itemize}

These questions influence both the qualitative research design and the engineering decisions described in later chapters.

\section{Project Objectives}

Based on the problem statement and research questions, the project has the following objectives:

\begin{itemize}
    \item \textbf{O1: Ecosystem understanding} --- Build a structured understanding of the rural after-sales ecosystem through primary research with consumers, technicians and dealers, and a review of existing platforms.
    \item \textbf{O2: Requirements and design} --- Formalise functional and non-functional requirements and propose a system design tailored to rural constraints, including architecture, data model and matching logic.
    \item \textbf{O3: MVP development} --- Implement a working full-stack MVP that supports end-to-end workflows from service request creation to job completion and feedback.
    \item \textbf{O4: Data and analytics} --- Instrument the MVP to log key events and derive basic operational metrics such as response time, completion rate and technician utilisation.
    \item \textbf{O5: Business and impact framing} --- Articulate a plausible business model, partner ecosystem and unit economics for scaling the platform beyond a purely academic prototype.
\end{itemize}

\section{Scope and Limitations}

The scope of this B.Tech project is intentionally constrained:

\begin{itemize}
    \item The primary focus is on rural and semi-urban households in a limited geographic region (for example, one or two districts).
    \item The appliance categories considered include ceiling fans, mixers, televisions and refrigerators, with smartphones treated as an adjacent but important category.
    \item The technical scope is a web-based MVP, optimised for mobile browsers; native mobile apps and fully offline modes are left to future work.
    \item Physical logistics such as spare-part supply chains, warranty reimbursement flows and technician transportation are modelled conceptually but not implemented.
\end{itemize}

Key limitations include the lack of a live, large-scale deployment during the project timeline, reliance on simulated pilots and dummy data for certain evaluations, and the absence of deep integration with brand CRM systems.

\section{Structure of the Report}

The remainder of this report is organised as follows:

\begin{itemize}
    \item Chapter~\ref{chap:context} analyses the rural after-sales ecosystem, stakeholder landscape and pain points emerging from research.
    \item Chapter~\ref{chap:literature} reviews related work, including existing service aggregators and rural technology interventions, and positions LastMile Service within this landscape.
    \item Chapter~\ref{chap:requirements} formalises user personas, functional and non-functional requirements, and high-level usage scenarios.
    \item Chapter~\ref{chap:design} describes the overall system architecture, data model, matching and reputation mechanisms, and key design choices.
    \item Chapter~\ref{chap:implementation} details the implementation of the LastMile Service MVP, including technology stack, APIs and front-end workflows.
    \item Chapter~\ref{chap:business} presents the business model, potential brand partnerships, partner mapping and illustrative unit economics.
    \item Chapter~\ref{chap:evaluation} discusses evaluation strategy, analytics, pilot simulations and key learnings.
    \item Chapter~\ref{chap:conclusion} concludes the report, summarises contributions and outlines future work.
\end{itemize}

%=====================================================
\chapter{Ecosystem and Problem Context}
\label{chap:context}

\section{Overview of the Rural After-Sales Ecosystem}

The ecosystem around appliance service in rural and semi-urban India involves multiple actors: end consumers, local technicians, multi-brand retailers, authorised service centres and large appliance brands.
Their incentives are not always aligned.
Consumers want quick, affordable and trustworthy service.
Technicians want stable income, access to spare parts and recognition.
Brands want to protect their reputation and minimise warranty leakage.
Retailers care about post-sales satisfaction to maintain repeat sales and preserve local goodwill.

In practice, the network is opaque.
A household often relies on a nearby shopkeeper or a neighbour's recommendation to find a technician.
Technicians may work informally with multiple shops or small franchisees but remain invisible to brand headquarters.
Brands maintain computerised systems and customer-care numbers, but these systems rarely include village-level actors, and outreach is often limited to district headquarters.

\section{Appliance Penetration Trends in Rural India}

Multiple government and industry reports over the last decade have documented a steady increase in appliance penetration in rural India, especially for fans, televisions, basic kitchen appliances and smartphones.
Electrification schemes and improved rural incomes have played a major role.
While exact numbers vary across states, three broad trends are clear:

\begin{itemize}
    \item The \textbf{ownership gap} between urban and rural households is narrowing for basic appliances such as fans and televisions.
    \item The \textbf{usage intensity} of smartphones has risen sharply, making them both a target for service interventions and a channel through which digital platforms can reach rural users.
    \item Appliance purchases are increasingly \textbf{aspirational}, with households stretching budgets for branded products, which raises expectations around after-sales service.
\end{itemize}

Our own survey (described later) is consistent with this picture: almost every respondent reported owning at least one fan and one smartphone, and a significant share reported having televisions and basic kitchen appliances such as mixers or grinders.
These ownership patterns directly shape the mix of repair and maintenance needs that a platform like LastMile Service must support in its early stages.

\section{Stakeholder Mapping}

For LastMile Service, four primary stakeholder groups are especially relevant.

\subsection*{Rural Consumers}

Typically, rural consumers are households with one to five key appliances.
Cash flows are often irregular, and the opportunity cost of losing a day's work to travel for service is high.
Connectivity may be intermittent, and phones are sometimes shared within the household.
Consumers prefer known local faces but worry about being overcharged or receiving poor-quality repairs.
Their relationship with brands is often indirect, mediated by local dealers.

\subsection*{Local Technicians}

Local technicians operate within a radius of 5--20 km.
Some have years of experience but no formal certification; others may be semi-skilled.
They rely on word-of-mouth referrals and shopkeepers for work.
Income is volatile, and access to genuine spare parts is constrained by geography and working capital.
Most technicians own a basic smartphone and use messaging apps, but complex digital workflows are rare in their day-to-day work.

\subsection*{Retailers and Dealers}

Multi-brand shops and small dealerships act as the face of appliance brands in rural markets.
They sell products, manage basic complaints and often recommend or directly arrange technicians.
To protect their reputation, they care about whether customers are satisfied long after the sale.
However, they typically lack tools to monitor service quality or track issues systematically across time.

\subsection*{Brand Service Networks}

Appliance and mobile brands operate call centres, warranty systems and authorised service centres.
These systems are optimised for cities and large towns.
In rural areas, brands rely on sparse service centres and ad-hoc arrangements with local workshops.
They have limited visibility into what actually happens once a product leaves the dealership, especially when customers consult informal technicians.

\subsection*{Platform Operator}

The final stakeholder is the platform operator of LastMile Service itself.
The operator must balance user experience, data security, technician fairness, brand expectations and financial viability.
In the context of this B.Tech project, the platform operator is conceptual --- represented by the project team and, in future, a potential startup or partner organisation.

\section{Current Repair Practices and Their Limitations}

The current repair journey for a rural household typically begins when an appliance fails.
The household may take one of several routes:

\begin{itemize}
    \item \textbf{Local shop route:} Visit the shop where the appliance was bought and request help.
    The shopkeeper may recommend an informal technician or a distant authorised centre.
    \item \textbf{Informal technician route:} Call a known technician directly or ask neighbours for recommendations.
    \item \textbf{Brand route:} For higher-value products under warranty, the household may attempt to contact the brand's toll-free number or website, often with varying success due to language barriers or connectivity.
\end{itemize}

To quantify these patterns, we conducted a Google Form survey with $N = 58$ respondents (mix of rural and semi-urban households and a small number of technicians).
Table~\ref{tab:survey_combined} summarises selected findings related to whom people contact for repairs, what problems they face and how they prefer to book service.

\begin{table}[H]
    \centering
    \small
    \caption{Combined key survey results (N = 58 respondents).}
    \label{tab:survey_combined}
    \begin{tabular}{|p{4cm}|p{5.2cm}|c|c|p{3cm}|}
        \hline
        \textbf{Question Group} & \textbf{Option / Response} & \textbf{Count} & \textbf{Percentage} & \textbf{Notes} \\
        \hline
        Whom do people usually contact for repairs? 
            & Local technician (nearby)           & 14 & 24.1\% & Single choice \\
            & Family / friend who repairs         & 13 & 22.4\% &               \\
            & Shop in the nearest town            & 12 & 20.7\% &               \\
            & Depends on the problem              & 12 & 20.7\% &               \\
            & Authorised brand service centre     &  7 & 12.1\% &               \\
            & \textbf{Total}                      & 58 & 100\%  &               \\
        \hline
        Problems faced during repair
            & No trained technician in village    & 37 & 63.8\% & Multiple choice \\
            & Travel or transport cost is high    & 35 & 60.3\% &                 \\
            & Too expensive                       & 35 & 60.3\% &                 \\
            & Parts not available                 & 34 & 58.6\% &                 \\
            & Do not know whom to trust           & 33 & 56.9\% &                 \\
            & Technicians come late               & 32 & 55.2\% &                 \\
            & Service centre is too far           & 29 & 50.0\% &                 \\
        \hline
        Preferred booking / contact methods
            & Local shop or CSC centre            & 30 & 51.7\% & Multiple choice \\
            & WhatsApp message                    & 28 & 48.3\% &                 \\
            & Social media groups                 & 28 & 48.3\% &                 \\
            & Phone call                          & 25 & 43.1\% &                 \\
            & Mobile app                          & 25 & 43.1\% &                 \\
            & In-person only                      & 25 & 43.1\% &                 \\
        \hline
        Willingness to use trained technician within 5--7 km
            & Yes                                 & 20 & 34.5\% & Single choice   \\
            & Maybe                               & 20 & 34.5\% &                 \\
            & No                                  & 18 & 31.0\% &                 \\
            & \textbf{Total}                      & 58 & 100\%  &                 \\
        \hline
    \end{tabular}
\end{table}

The table shows that households heavily rely on nearby local technicians and informal contacts, with authorised brand centres accounting for only a small fraction of first contacts.
At the same time, major pain points relate to lack of trained technicians, travel costs, uncertainty about whom to trust and spare-part availability.
Preferences for booking methods are diverse, with local touchpoints, phone and WhatsApp all playing important roles --- an important design input for LastMile.

Figures~\ref{fig:pref_repair}, \ref{fig:repair_replace_customers} and \ref{fig:repair_replace_techs} further illustrate attitudes towards where to go for service and whether to repair or replace faulty appliances.

\begin{figure}[H]
    \centering
    \begin{subfigure}[b]{0.48\textwidth}
        \includegraphics[width=\textwidth]{img5}
        \caption{Bar chart}
    \end{subfigure}
    \hfill
    \begin{subfigure}[b]{0.48\textwidth}
        \includegraphics[width=\textwidth]{img6}
        \caption{Pie chart}
    \end{subfigure}
    \caption{Preferred repair service provider among surveyed customers.}
    \label{fig:pref_repair}
\end{figure}

As seen in Figure~\ref{fig:pref_repair}, a majority of respondents prefer brand service centres, but a substantial minority continues to rely on local technicians, and many respondents answer ``mixed/depends''.
This confirms that any new platform must work \emph{with} local technicians and brand networks rather than trying to replace one with the other.

% \begin{figure}[H]
%     \centering
%     \begin{subfigure}[b]{0.48\textwidth}
%         \includegraphics[width=\textwidth]{img7}
%         \caption{Bar chart}
%     \end{subfigure}
%     \hfill
%     \begin{subfigure}[b]{0.48\textwidth}
%         \includegraphics[width=\textwidth]{img8}
%         \caption{Pie chart}
%     \end{subfigure}
%     \caption{Customer attitudes towards repairing versus replacing faulty appliances.}
%     \label{fig:repair_replace_customers}
% \end{figure}

% \begin{figure}[H]
%     \centering
%     \begin{subfigure}[b]{0.48\textwidth}
%         \includegraphics[width=\textwidth]{img9}
%         \caption{Bar chart}
%     \end{subfigure}
%     \hfill
%     \begin{subfigure}[b]{0.48\textwidth}
%         \includegraphics[width=\textwidth]{img10}
%         \caption{Pie chart}
%     \end{subfigure}
%     \caption{Technician attitudes towards repairing versus replacing appliances.}
%     \label{fig:repair_replace_techs}
% \end{figure}

Figures~\ref{fig:repair_replace_customers} and~\ref{fig:repair_replace_techs} suggest that most customers prefer to try repair first or decide based on cost, while technicians overwhelmingly favour repair over replacement.
This alignment creates a favourable context for a service platform that can make repairs more reliable, transparent and convenient.

\section{Identified Pain Points and Opportunity Areas}

Through field conversations and secondary research, several recurring pain points emerge:

\begin{itemize}
    \item \textbf{Discovery and trust:} Households do not know which technician is trustworthy or qualified for a specific brand or appliance type.
    \item \textbf{Transparency of costs and quality:} There is little clarity about expected charges, spare-part provenance or the impact of third-party repairs on warranty.
    \item \textbf{Travel and time costs:} For brand-authorised service, customers often travel long distances, losing both time and income.
    \item \textbf{Technician livelihood volatility:} Skilled technicians experience uneven demand and depend heavily on a small set of intermediaries.
    \item \textbf{Brand visibility:} Brands have limited data on rural failure patterns, service resolutions and customer satisfaction.
\end{itemize}

These qualitative insights are supported by the survey results on pain points, visualised in Figure~\ref{fig:painpoints}.

% \begin{figure}[H]
%     \centering
%     \begin{subfigure}[b]{0.48\textwidth}
%         \includegraphics[width=\textwidth]{img13}
%         \caption{Bar chart}
%     \end{subfigure}
%     \hfill
%     \begin{subfigure}[b]{0.48\textwidth}
%         \includegraphics[width=\textwidth]{img14}
%         \caption{Pie chart}
%     \end{subfigure}
%     \caption{Pain points mentioned in interviews and survey responses.}
%     \label{fig:painpoints}
% \end{figure}

Quality and trust issues, spare-part availability and transparency around timing and costs emerge as the most frequently cited concerns, followed by technician availability and low earnings for technicians themselves.
These pain points translate directly into opportunity areas for LastMile Service: verified technician profiles, standardised pricing slabs, support for spare-part logistics, transparent status updates and mechanisms to improve technician livelihoods.

Table~\ref{tab:painpoints} summarises how these issues map to different stakeholders and outlines the corresponding opportunity space.

\begin{table}[H]
    \centering
    \caption{Summary of key pain points and opportunity areas in rural after-sales service.}
    \label{tab:painpoints}
    \begin{tabular}{p{3.2cm}p{3cm}p{7cm}}
        \toprule
        \textbf{Pain Point} & \textbf{Primary Stakeholder} & \textbf{Opportunity for LastMile Service} \\
        \midrule
        Lack of trusted technician discovery & Consumer & Provide verified technician profiles, ratings and brand tags. \\
        Unclear pricing and quality & Consumer / Technician & Standardise pricing slabs; record job histories; enable feedback loops. \\
        Long travel to service centres & Consumer & Enable doorstep service by local technicians linked to brands. \\
        Volatile technician income & Technician & Offer more predictable job flow and performance-based incentives. \\
        Poor visibility into rural repairs & Brand & Aggregate anonymised data on failures, resolutions and satisfaction. \\
        \bottomrule
    \end{tabular}
\end{table}
%=====================================================
\chapter{Survey Methodology}
\label{chap:survey_methodology}

\section{Overview}

To ground the design of LastMile Service in real user needs, we conducted a structured household survey focused on appliance ownership and repair experiences.
The goal was to understand how rural and semi-urban households currently handle breakdowns of home appliances and mobile phones, which channels they use for repair, what problems they face in the process, and how open they are to a new platform that connects them to trained local technicians.

The survey specifically targeted households that own at least one major appliance or smartphone and have had at least one repair experience in the recent past.
Responses were collected using a Google Form titled \emph{``Home Appliance \& Mobile Repair Experience Survey''}, shared via messaging apps and supplemented by phone-assisted filling where needed.

\section{Target Population and Sampling}

The target population for the survey consisted of rural and semi-urban households in India.
Links to the Google Form were circulated through personal networks, local contacts and student families, with an emphasis on households located outside major metropolitan areas.
Where respondents were not comfortable filling the form themselves, their answers were recorded over a phone call by the project team.

The survey does not claim to be statistically representative at a national level.
Instead, it is used as an exploratory instrument to capture patterns, pain points and preferences that inform the requirements and design of the platform.

\section{Questionnaire Design}

The questionnaire comprised multiple-choice and short-answer questions grouped into four broad sections:

\begin{itemize}
    \item \textbf{Household and location profile:} basic information such as state, district and settlement type (village, town).
    \item \textbf{Appliance ownership and repair history:} types of appliances owned, which devices had required repair in the last year and how those repairs were handled.
    \item \textbf{Repair journey and pain points:} whom respondents usually contact for repairs, distance and time to service, and common problems faced during the process.
    \item \textbf{Digital access and openness to a platform:} smartphone availability, preferred channels for booking service, and willingness to use a platform that connects them to trained local technicians.
\end{itemize}

Most questions were closed-ended to allow basic quantitative analysis, with a few open-ended questions inviting respondents to describe specific bad experiences or suggestions in their own words.

\section{Data Collection Procedure}

The survey was administered over a defined period using a Google Form.
The link was shared via WhatsApp and other messaging channels to contacts residing in rural and semi-urban areas.
In some cases, the project team conducted phone calls and filled the form on behalf of respondents who were less comfortable with online forms.

Duplicate or clearly incomplete responses were removed during cleaning.
The final dataset, summarised in Table~\ref{tab:survey_sample}, forms the basis for the descriptive statistics and charts used in the ecosystem, requirements and evaluation chapters of this report.

\section{Sample Summary}

Table~\ref{tab:survey_sample} provides a concise summary of the survey sample.

\begin{table}[H]
    \centering
    \caption{Summary of household survey sample (N = 58).}
    \label{tab:survey_sample}
    \begin{tabular}{p{5cm}p{8cm}}
        \toprule
        \textbf{Metric} & \textbf{Value} \\
        \midrule
        Number of responses & \textbf{58} complete household responses after basic cleaning. \\
        States / districts covered & Multiple districts across India, with a concentration in small towns and rural blocks in North India. \\
        Data collection mode & Online Google Form, supplemented by phone-assisted entries for respondents less comfortable with digital forms. \\
        Settlement type & Mix of rural villages and semi-urban towns, with a majority of responses coming from rural or peri-urban locations. \\
        Data collection period & Late 2025 (approximately four weeks from launch of the Google Form). \\
        \bottomrule
    \end{tabular}
\end{table}

\section{Selected Quantitative Findings}

While the survey is exploratory, several quantitative patterns are directly relevant for the design of LastMile Service.
This section highlights a subset of closed-ended questions; detailed counts are given in Table~\ref{tab:combined_survey_results}, and corresponding visualisations are shown in Figures~\ref{fig:service_pref_customers}--\ref{fig:desired_features_all}.

\subsection*{Whom do people usually contact for repairs?}

Respondents most frequently rely on nearby local technicians, family or friends who know how to repair, and shops in the nearest town.
Authorised brand service centres account for a much smaller share of first contacts.

\begin{figure}[H]
    \centering
    \begin{subfigure}{0.49\textwidth}
        \centering
        \includegraphics[width=\textwidth]{img5}
        \caption{Bar chart.}
    \end{subfigure}
    \begin{subfigure}{0.49\textwidth}
        \centering
        \includegraphics[width=\textwidth]{img6}
        \caption{Pie chart.}
    \end{subfigure}
    \caption{Preferred repair service among customers (brand centre vs.\ local technician vs.\ mixed).}
    \label{fig:service_pref_customers}
\end{figure}

\subsection*{Repair versus replacement tendencies}

For most common appliances, both customers and technicians exhibit a strong \emph{repair-first} mindset, with replacement considered mainly when devices are very old or repair costs are unusually high.

% \begin{figure}[H]
%     \centering
%     \begin{subfigure}{0.49\textwidth}
%         \centering
%         \includegraphics[width=\textwidth]{img7}
%         \caption{Customers (bar).}
%     \end{subfigure}
%     \begin{subfigure}{0.49\textwidth}
%         \centering
%         \includegraphics[width=\textwidth]{img8}
%         \caption{Customers (pie).}
%     \end{subfigure}
%     \caption{Customer preferences on repair vs.\ replace decisions.}
%     \label{fig:repair_vs_replace_customers}
% \end{figure}

\begin{figure}[H]
    \centering
    \begin{subfigure}{0.49\textwidth}
        \centering
        \includegraphics[width=\textwidth]{img9}
        \caption{Technicians (bar).}
    \end{subfigure}
    \begin{subfigure}{0.49\textwidth}
        \centering
        \includegraphics[width=\textwidth]{img10}
        \caption{Technicians (pie).}
    \end{subfigure}
    \caption{Technician perspectives on repair vs.\ replace decisions.}
    \label{fig:repair_vs_replace_technicians}
\end{figure}

\subsection*{Openness to a digital platform}

Across the combined sample, a clear majority is positive or strongly positive about the idea of a digital platform that connects them to trained technicians, with only a small neutral or uncertain segment.

% \begin{figure}[H]
%     \centering
%     \begin{subfigure}{0.49\textwidth}
%         \centering
%         \includegraphics[width=\textwidth]{img11}
%         \caption{Bar chart.}
%     \end{subfigure}
%     \begin{subfigure}{0.49\textwidth}
%         \centering
%         \includegraphics[width=\textwidth]{img12}
%         \caption{Pie chart.}
%     \end{subfigure}
%     \caption{Support for the proposed digital platform idea across all respondents.}
%     \label{fig:platform_support}
% \end{figure}

\subsection*{Pain points and desired features}

Pain points related to quality and trust, spare parts availability, transparency of timing and earnings for technicians emerged strongly (see Figure~\ref{fig:painpoints_all} in Chapter~\ref{chap:context}).
When asked about desired platform features, respondents emphasised ratings and verification, live tracking and estimated time of arrival (ETA), spare-part support and home pickup/service (Figure~\ref{fig:desired_features_all}).

\begin{figure}[H]
    \centering
    \begin{subfigure}{0.49\textwidth}
        \centering
        \includegraphics[width=\textwidth]{img15}
        \caption{Bar chart.}
    \end{subfigure}
    \begin{subfigure}{0.49\textwidth}
        \centering
        \includegraphics[width=\textwidth]{img16}
        \caption{Pie chart.}
    \end{subfigure}
    \caption{Desired platform features mentioned by survey respondents.}
    \label{fig:desired_features_all}
\end{figure}

\subsection*{Aggregated key results}

Table~\ref{tab:combined_survey_results} consolidates a few of the most decision-relevant questions, combining counts and percentages for quick reference.
These results directly inform the functional requirements and value propositions of LastMile Service.

\begin{table}[H]
\centering
\small
\caption{Selected key survey results (N = 58).}
\label{tab:combined_survey_results}
\begin{tabular}{p{4cm}p{5.2cm}ccp{3cm}}
\toprule
\textbf{Question group} & \textbf{Option / response} & \textbf{Count} & \textbf{Percentage} & \textbf{Notes} \\
\midrule
Whom do people usually contact for repairs? 
    & Local technician (nearby)           & 14 & 24.1\% & Single choice \\
    & Family / friend who repairs         & 13 & 22.4\% &                \\
    & Shop in the nearest town            & 12 & 20.7\% &                \\
    & Depends on the problem              & 12 & 20.7\% &                \\
    & Authorised brand service centre     &  7 & 12.1\% &                \\
    & \textbf{Total}                      & 58 & 100\%  &                \\
\midrule
Problems faced during repair
    & No trained technician in village    & 37 & 63.8\% & Multiple choice \\
    & Travel or transport cost is high    & 35 & 60.3\% &                 \\
    & Too expensive                       & 35 & 60.3\% &                 \\
    & Parts not available                 & 34 & 58.6\% &                 \\
    & Do not know whom to trust           & 33 & 56.9\% &                 \\
    & Technicians come late               & 32 & 55.2\% &                 \\
    & Service centre is too far           & 29 & 50.0\% &                 \\
\midrule
Preferred booking / contact methods
    & Local shop or CSC centre            & 30 & 51.7\% & Multiple choice \\
    & WhatsApp message                    & 28 & 48.3\% &                 \\
    & Social media groups                 & 28 & 48.3\% &                 \\
    & Phone call                          & 25 & 43.1\% &                 \\
    & Mobile app                          & 25 & 43.1\% &                 \\
    & In-person only                      & 25 & 43.1\% &                 \\
\midrule
Willingness to use trained technician within 5--7 km
    & Yes                                 & 20 & 34.5\% & Single choice   \\
    & Maybe                               & 20 & 34.5\% &                 \\
    & No                                  & 18 & 31.0\% &                 \\
    & \textbf{Total}                      & 58 & 100\%  &                 \\
\bottomrule
\end{tabular}
\end{table}

These observations feed directly into the requirements, personas and matching logic developed in the next chapters, and later into possible machine-learning extensions such as demand prediction or intelligent routing.

%=====================================================
\chapter{Related Work and Market Landscape}
\label{chap:literature}

\section{Existing Service Platforms}

Several digital platforms currently operate in the appliance and device service domain.
Examples include Urban Company, OnsiteGo, Servify and various brand-specific service apps.
They typically focus on urban and metropolitan markets, providing on-demand booking, subscription plans and extended warranties.
Their design assumes reliable internet connectivity, digital payments and a dense network of service professionals.

From the perspective of LastMile Service, these platforms serve as useful reference points for:
\begin{itemize}[noitemsep]
    \item how users express service needs digitally (issue categories, photos, preferred time slots);
    \item how technician profiles, ratings, certifications and pricing models are presented;
    \item how customer support and escalation flows are structured once a job is booked.
\end{itemize}

However, their rural presence is limited, and their onboarding processes are not tuned for small-town technicians with lower digital literacy, intermittent connectivity or cash-based transactions.

\section{Rural Technology Interventions}

Beyond appliance service, various rural technology interventions have experimented with hybrid models that combine local actors with digital backbones.
Examples include agricultural advisory platforms, telemedicine pilots and rural e-commerce initiatives.
Common themes across these efforts include:
\begin{itemize}[noitemsep]
    \item leveraging local intermediaries (e.g., village-level entrepreneurs) to bridge digital gaps;
    \item designing interfaces that minimise typing and rely on voice, icons or structured options;
    \item handling intermittent connectivity through offline-first or asynchronous workflows.
\end{itemize}

LastMile Service draws inspiration from these patterns, treating local technicians and dealers as crucial intermediaries rather than bypassing them.

\section{Competitive and Complementary Players}

The broader competitive and complementary landscape for LastMile Service includes:
\begin{itemize}[noitemsep]
    \item \textbf{Urban-focused aggregators:} Urban Company and similar platforms, which could extend into rural areas if they had a suitable technician network and appropriate economics.
    \item \textbf{Device-care and protection startups:} Platforms such as OnsiteGo and Servify, which partner with brands to offer extended warranties, protection plans and replacement services.
    \item \textbf{Brand service apps:} Official apps from large appliance and handset brands that allow registration of products and booking of service complaints, typically limited to a single-brand ecosystem.
\end{itemize}

Rather than compete head-on with all of these players, LastMile Service positions itself as a rural-first, technician-centric layer that can potentially integrate with or feed into multiple upstream systems.

\section{Gap Analysis}

A simplified gap analysis reveals that:
\begin{itemize}[noitemsep]
    \item there is no widely adopted rural-first digital platform dedicated specifically to appliance service (as opposed to generic home services);
    \item local technicians are rarely integrated into brand ecosystems in a structured, data-driven manner;
    \item reputation systems, where they exist, are designed for highly online users and are not optimised for low literacy or intermittent connectivity;
    \item brands face a blind spot when it comes to rural failures and repairs undertaken outside their official networks.
\end{itemize}

These gaps substantiate the relevance of a project like LastMile Service and help narrow its focus.

\section{Positioning of LastMile Service}

Within this landscape, LastMile Service positions itself as a \emph{trust-based, brand-aligned marketplace} with three main differentiators:
\begin{itemize}[noitemsep]
    \item a primary focus on semi-urban and rural households, where constraints differ sharply from metros;
    \item an explicit emphasis on onboarding, supporting and gradually upskilling local technicians, including mechanisms for brand-linked certification;
    \item an architecture that can surface aggregated data and insights to brands while retaining the flexibility to support multiple appliances, geographies and future machine-learning modules.
\end{itemize}

Figure~\ref{fig:lastmile_overview} in Chapter~\ref{chap:context} visually summarises this positioning, showing how rural households, communication channels and brand partners are connected through the LastMile Service platform.

The remainder of the report translates this positioning into concrete requirements, design, implementation, evaluation and, in future work, data pipelines and machine-learning components such as recommendation models and support chatbots.

%=====================================================
\chapter{Requirements and User Personas}
\label{chap:requirements}

\section{User Personas}

To ground the design in reality, we formalise four user personas derived from interviews, survey responses and secondary research.
The core entities and flows illustrated in Figure~\ref{fig:entity_flow} (Chapter~\ref{chap:intro}) influenced how these personas interact with the system.

\subsection*{P1: Rural Homemaker (Consumer)}

The rural homemaker is often the primary decision-maker for day-to-day appliance usage.
She manages multiple responsibilities and may not always have the time or means to travel to distant service centres.
Digital comfort is moderate: she can use WhatsApp and make calls but may not be comfortable with long text forms or complex apps.
She values punctuality, clear prices and the reassurance of dealing with a known or vetted technician.

\subsection*{P2: Local Technician}

The local technician owns basic tools, travels by bike and serves multiple villages.
He may have learnt the trade through apprenticeship rather than formal training.
Digital comfort includes using smartphones for calls, messaging and simple apps, but not necessarily complex dashboards.
He wants more predictable job flow, fewer unpaid trips and access to genuine parts at reasonable cost.
He is cautious about platforms that might reduce his autonomy or earnings.

\subsection*{P3: Shop Owner / Dealer}

The shop owner sells appliances from multiple brands and operates in a tightly knit local market where reputation matters.
He currently acts as an informal bridge between consumers and technicians, maintaining a mental list of whom to call for which problem.
He would like better visibility into whether service requests were handled satisfactorily and how they reflect on his shop and brand relationships.

\subsection*{P4: Brand Service Manager}

The brand service manager works at regional or national level and is responsible for customer satisfaction, warranty adherence and service KPI targets.
She is used to dashboards, CRM systems and formal reports.
Her ideal outcome from a platform like LastMile Service is better visibility into rural service activity and technician quality without having to build a dispersed field team from scratch.

\section{Business Model Canvas}

Figure~\ref{fig:bmc} summarises the overall business logic of LastMile Service
using the Business Model Canvas framework. It captures, on a single page, the
key partners, activities, resources, value propositions, customer segments and
revenue/cost structure discussed in this chapter.

\begin{figure}[H]
    \centering
    \includegraphics[width=\textwidth]{img26.png} % <-- your BMC image file
    \caption{Business Model Canvas for LastMile Service.}
    \label{fig:bmc}
\end{figure}

\section{Functional Requirements}

Combining insights from personas and the problem context, the key functional requirements are:
\begin{enumerate}[label=F\arabic*.,noitemsep]
    \item User registration and authentication for consumers, technicians, dealers and admins.
    \item Creation of service requests specifying appliance type, brand, issue description and preferred time slot.
    \item Discovery of available technicians based on location, skills, brand-certification tags and ratings.
    \item Automated or semi-automated assignment of technicians to service requests.
    \item Real-time status updates for jobs: created, accepted, in-progress, completed, cancelled.
    \item Notifications to users (via SMS, WhatsApp or in-app) for key events such as assignment, rescheduling and completion.
    \item Capture of payment details and basic invoice generation, with flexibility for cash-based transactions.
    \item Feedback and rating mechanisms for completed jobs, visible to future consumers and the platform.
    \item Admin and operator dashboards to monitor jobs, technicians and basic analytics.
\end{enumerate}

\section{Non-functional Requirements}

Non-functional requirements are equally important in rural settings:
\begin{itemize}[noitemsep]
    \item \textbf{Usability:} Interfaces must be simple, using clear icons, short labels and minimal free-text input.
    \item \textbf{Performance:} The platform should be usable on low-end Android devices over 3G or unstable 4G connections.
    \item \textbf{Reliability:} Critical flows (job creation, status updates) must be robust against dropped connections and retries.
    \item \textbf{Security and privacy:} Phone numbers, addresses and other personally identifiable information should be protected and visible only to authorised parties.
    \item \textbf{Scalability:} The architecture should support gradual growth in users and jobs without major rework.
\end{itemize}

\section{Usage Scenarios}

Representative usage scenarios include:
\begin{itemize}[noitemsep]
    \item A consumer uses her smartphone to raise a service request for a non-working fan, receives a technician assignment, tracks status updates and then rates the service after completion.
    \item A technician logs in each morning, sees a list of assigned jobs with locations and issues, updates statuses through the day and reviews his rating trends over time.
    \item An admin monitors all open jobs in a district and intervenes if a request remains unassigned or unaccepted beyond a threshold time.
    \item A brand service manager periodically reviews aggregated dashboards on rural service volume, response times and satisfaction scores to decide where to invest in training or additional technician capacity.
\end{itemize}

These scenarios inform the design of user flows, screens and, later, the data captured for analytics and machine-learning models.

%=====================================================
\chapter{System Design}
\label{chap:design}

\section{Architecture Overview}

The LastMile Service system follows a multi-tier architecture with separate layers for presentation, application logic and data storage.
While the deployed MVP is implemented as a modular monolith, the design is guided by microservices-style patterns so that individual modules can be split out in future.

The high-level architecture, inspired by standard microservices patterns, is illustrated in Figure~\ref{fig:system_architecture}.

% \begin{figure}[H]
%     \centering
%     \includegraphics[width=\textwidth]{img3}
%     \caption{Microservices-style patterns (API gateway, BFF, CQRS, event sourcing and saga orchestration) that inform the architecture of the LastMile Service backend.}
%     \label{fig:system_architecture}
% \end{figure}

The main components are:
\begin{itemize}[noitemsep]
    \item \textbf{Frontend:} A responsive web application (e.g., React with Vite / Next.js) serving consumers, technicians, dealers and admins through role-based views.
    \item \textbf{Backend API:} A RESTful API implemented using Node.js and Express that encapsulates business logic, handles validation, authentication and authorisation, and exposes endpoints for mobile/desktop clients.
    \item \textbf{Database:} A document or relational database (e.g., MongoDB or PostgreSQL) storing users, technician profiles, service requests, reviews and audit logs.
    \item \textbf{Notification Services:} Integration with SMS / WhatsApp gateways and email providers for transactional notifications.
    \item \textbf{Analytics and ML Layer (future-ready):} A separate store or pipeline that consumes event logs from the backend (job creation, status transitions, ratings) to support dashboards, technician-performance analytics and, in future work, machine-learning models for matching and a support chatbot.
\end{itemize}

\section{Key Modules}

\subsection{Authentication and User Management}

This module provides registration and login flows for consumers, technicians, dealers and admins.
Technicians may require additional verification steps (for example, uploading ID proof or declaring skills and brands handled).
Authentication is handled using JSON Web Tokens (JWTs), and role-based access control restricts sensitive operations such as technician approval and manual job reassignment.

\subsection{Service Request Management}

Consumers can create new service requests by specifying appliance category, brand, a short issue description and location (address plus coarse coordinates where available).
Requests are stored with a unique identifier and an initial status of \texttt{Created}.
Optional fields such as photos or preferred time slots can be added to improve technician preparation and later analytics.

\subsection{Technician Discovery and Matching}

When a request is created, the system identifies candidate technicians based on:
\begin{itemize}[noitemsep]
    \item service radius around the customer's location;
    \item skill tags for appliance categories;
    \item brand-certification tags, where available;
    \item current workload and recent reliability metrics.
\end{itemize}

A scoring function (Section~\ref{sec:matching_reputation}) combines these factors to rank technicians.
Depending on configuration, the system may auto-assign the top-ranked technician or allow technicians to accept requests from a pool.
All decisions and candidate lists are logged so that they can later be used to train a data-driven model.

\subsection{Job Tracking and Feedback}

Throughout the job lifecycle, technicians update status transitions such as \texttt{Accepted}, \texttt{EnRoute}, \texttt{InProgress} and \texttt{Completed}.
Consumers see simplified status labels and can contact the technician if needed (for example, by phone or WhatsApp).
Upon completion, the consumer can rate the experience and optionally leave a short review, contributing to the technician's reputation and providing training data for future recommendation models.


\section{Data Model}

Figure~\ref{fig:erd} summarises the logical data model underlying the MVP at a conceptual level.

\begin{figure}[H]
    \centering
    \includegraphics[width=0.65\textwidth]{img}
    \caption{Simplified relationship between key entities in the LastMile Service MVP (service requests, technician profiles and reviews).}
    \label{fig:erd}
\end{figure}

The core entities are:

\begin{itemize}[noitemsep]
    \item \texttt{User}: Base record for all users, including role (consumer, technician, admin, dealer), contact details and authentication metadata.
    \item \texttt{TechnicianProfile}: Technician-specific details such as skills (appliance categories), service radius, brand tags, verification status and aggregate rating.
    \item \texttt{ServiceRequest}: Records individual service needs with appliance information, issue description, status, timestamps and links to the requesting consumer and assigned technician.
    \item \texttt{Review}: Stores consumer feedback (rating plus short text) after job completion.
    \item \texttt{JobStatusHistory}: An audit log capturing all status transitions for a given job, used both for debugging and for computing KPIs such as response and resolution time.
\end{itemize}


%=====================================================
\chapter{Implementation}
\label{chap:implementation}

\section{Technology Stack}

The MVP is implemented using the following stack:

\begin{itemize}
    \item \textbf{Frontend:} React (with Vite or Next.js) for component-based UI and routing; responsive design ensures usability on mobile browsers.
    \item \textbf{Backend:} Node.js with Express.js for REST APIs and business logic.
    \item \textbf{Database:} MongoDB Atlas or a similar managed instance for rapid iteration and flexible schema evolution.
    \item \textbf{Styling:} Tailwind CSS or CSS modules for consistent styling and responsive layouts.
    \item \textbf{Deployment:} Frontend deployed on a service such as Vercel/Netlify; backend deployed on Render/Railway or a virtual private server.
    \item \textbf{Data and ML (future work):} Python-based notebooks (Pandas, scikit-learn / PyTorch) planned for training recommendation models and support chatbots on anonymised job and feedback logs.
\end{itemize}

\section{Backend Implementation}

The backend exposes RESTful endpoints grouped by domain: authentication, users, service requests, technicians and reviews.
Example routes include:

\begin{itemize}
    \item \verb|POST /api/auth/register|, \verb|POST /api/auth/login|
    \item \verb|POST /api/requests| to create a new service request
    \item \verb|GET /api/requests/:id| to fetch request details
    \item \verb|POST /api/requests/:id/assign| to assign a technician
    \item \verb|POST /api/requests/:id/status| to update job status
    \item \verb|POST /api/reviews| to submit feedback
\end{itemize}

Middleware layers enforce authentication, validate request bodies and handle errors in a consistent JSON format.
Database interactions are encapsulated using models or repositories, reducing coupling between routes and persistence details.
Event logs for each status transition are written to a separate collection so that they can later be used to compute KPIs and train machine-learning models.

\section{Frontend Implementation}

The frontend is organised into pages and reusable components:

\begin{itemize}
    \item \textbf{Landing and onboarding:} Introduces the platform, explains the roles and guides users to sign-up or login.
    \item \textbf{Consumer dashboard:} Shows active and past service requests, with a clear call-to-action for creating a new request.
    \item \textbf{Technician dashboard:} Lists assigned jobs, their statuses and locations, with simple controls for updating progress.
    \item \textbf{Admin view:} Provides an overview of all jobs, filters by status and district, and access to technician management tools.
\end{itemize}

Forms balance expressiveness with simplicity; they rely heavily on dropdowns and fixed options to reduce typing.
The layouts follow a “card + list” pattern so that the same UI works well on both desktop and small-screen Android devices.

Representative screens from the current frontend prototype are shown in Figures~\ref{fig:landing_ui} and~\ref{fig:technician_join_ui}.

\begin{figure}[H]
    \centering
    \includegraphics[width=\textwidth]{img18}
    \caption{Consumer-facing landing page of the LastMile Service MVP, highlighting the core value proposition and entry point for creating a service request.}
    \label{fig:landing_ui}
\end{figure}

\begin{figure}[H]
    \centering
    \includegraphics[width=\textwidth]{img24}
    \caption{Technician onboarding screen where local technicians can register, provide contact details and declare appliance skills.}
    \label{fig:technician_join_ui}
\end{figure}

Additional flows, such as browsing verified professionals (Figure~\ref{fig:browse_professionals}) and contacting support (Figure~\ref{fig:contact_ui}), are also implemented.

\begin{figure}[H]
    \centering
    \includegraphics[width=\textwidth]{img17}
    \caption{Browse-and-discover screen showing a list of service professionals with filters and basic pricing information.}
    \label{fig:browse_professionals}
\end{figure}

\begin{figure}[H]
    \centering
    \includegraphics[width=\textwidth]{img20}
    \caption{Contact and support form that allows users to submit detailed queries or escalations to the platform operator.}
    \label{fig:contact_ui}
\end{figure}

\section{Security and Privacy Considerations}

Even as an academic MVP, basic security and privacy practices are followed:

\begin{itemize}
    \item Passwords are hashed using a strong algorithm (e.g., bcrypt) before storage.
    \item JWT-based authentication ensures stateless session handling.
    \item Role checks are enforced at route level to prevent privilege escalation.
    \item Sensitive information such as exact addresses or phone numbers is only shown to parties who need it to complete the job.
    \item Logs used for analytics or future ML training are anonymised so that individual households and technicians cannot be re-identified.
\end{itemize}

%=====================================================
\chapter{Business Model and Partner Ecosystem}
\label{chap:business}

\section{Value Proposition}

\subsection*{For Consumers}

For rural consumers, LastMile Service offers:

\begin{itemize}
    \item A single, simple channel to request service without negotiating across multiple shops.
    \item Transparent, predictable pricing slabs for common issues.
    \item Visibility into the identity and reputation of the technician visiting their home.
\end{itemize}

\subsection*{For Technicians}

For local technicians, the platform provides:

\begin{itemize}
    \item Access to a larger and more predictable stream of jobs.
    \item A digital track record of completed work and ratings that can be used to negotiate better terms with brands or shops.
    \item Opportunities to participate in training modules and brand-certification programmes.
\end{itemize}

Figure~\ref{fig:tech_portal_screens} illustrates two representative technician-facing screens from the prototype.

\begin{figure}[H]
    \centering
    \begin{subfigure}[b]{0.48\textwidth}
        \centering
        \includegraphics[width=\textwidth]{img21}
        \caption{Certification workflow for uploading ID/ITI proof and completing a safety quiz.}
        \label{fig:certification_screen}
    \end{subfigure}\hfill
    \begin{subfigure}[b]{0.48\textwidth}
        \centering
        \includegraphics[width=\textwidth]{img23}
        \caption{Training programmes catalogue with brand-aligned modules for different appliance categories.}
        \label{fig:training_screen}
    \end{subfigure}
    \caption{Technician portal screens supporting verification and upskilling.}
    \label{fig:tech_portal_screens}
\end{figure}

\subsection*{For Brands and Partners}

For appliance and mobile brands, as well as device-care platforms, LastMile Service can:

\begin{itemize}
    \item Extend after-sales reach into rural areas without opening numerous physical centres.
    \item Provide structured data on failures, resolutions and customer satisfaction.
    \item Serve as a channel to roll out training, safety guidelines and documentation to the technician network.
\end{itemize}

For brand managers, a separate entry point is exposed (Figure~\ref{fig:schedule_demo}) so that they can request a demo and discuss custom integrations.

\begin{figure}[H]
    \centering
    \includegraphics[width=\textwidth]{img25}
    \caption{Brand-facing \emph{Schedule a Demo} screen for initiating partnerships and integrations.}
    \label{fig:schedule_demo}
\end{figure}

\section{Potential Partner Companies}

Based on preliminary mapping, partner categories include appliance brands, mobile brands and service/insurance platforms.
Table~\ref{tab:partners} summarises a few illustrative examples.

\begin{table}[H]
    \centering
    \caption{Illustrative partner companies and why they are relevant.}
    \label{tab:partners}
    \begin{tabular}{p{3cm}p{3cm}p{7cm}}
        \toprule
        \textbf{Company} & \textbf{Category} & \textbf{Relevance} \\
        \midrule
        LG Electronics & Appliance brand & Strong rural presence through fan and appliance sales; can benefit from structured rural service coverage. \\
        Xiaomi / Redmi & Mobile brand & Popular in semi-urban and rural markets; doorstep repair reduces the need for long travel. \\
        Vivo / Oppo & Mobile brand & High adoption in smaller towns; partnership can improve customer satisfaction where service centres are sparse. \\
        Havells / Whirlpool & Appliance brands & Large installed base of fans and white goods in rural households; improved after-sales can enhance brand loyalty. \\
        Urban Company, OnsiteGo, Servify & Platforms & Urban or pan-India service and protection players that could use a rural technician network for deeper reach. \\
        \bottomrule
    \end{tabular}
\end{table}

\section{Revenue Streams}

Possible revenue streams for LastMile Service include:

\begin{itemize}
    \item \textbf{Per-job commission:} A small platform fee on completed jobs, structured so that technicians still receive the majority of the visit fee.
    \item \textbf{Technician subscriptions:} Optional premium plans for technicians that unlock better visibility, priority in matching and access to training content.
    \item \textbf{Brand partnerships:} Annual contracts with brands for co-branded service programmes and access to aggregated analytics and ML-based insights (for example, failure hotspots, common part issues).
\end{itemize}

\section{Illustrative Unit Economics}

To build intuition, consider a typical fan repair.

\begin{itemize}
    \item A consumer pays a fixed visit fee plus any part cost (e.g., capacitor replacement).
    \item The technician receives most of the visit fee and may receive a small performance bonus for high ratings.
    \item The platform retains a modest commission to fund operations, support and continued product development, including analytics dashboards and future machine-learning components such as recommendation models and chatbots.
\end{itemize}

These assumptions can be expanded into detailed per-job contribution calculations, including one-time technician onboarding costs, expected job volumes and break-even analysis for different districts and appliance mixes.
%=====================================================
\chapter{Evaluation, Analytics and Learnings}
\label{chap:evaluation}

\section{Data Logging and Metrics}

The platform logs key events such as service request creation, assignment, status updates and review submissions.
From these logs, several operational metrics can be derived:

\begin{itemize}
    \item \textbf{Average response time:} time from request creation to technician acceptance.
    \item \textbf{Average resolution time:} time from acceptance to completion.
    \item \textbf{Job completion rate} and distribution of cancellation reasons.
    \item \textbf{Technician utilisation:} fraction of working hours associated with active jobs.
    \item \textbf{Rating metrics:} average rating, rating distribution and rating trajectory over time for technicians.
\end{itemize}

These event logs are stored in the same database as core entities, forming an implicit ``analytics layer'' that can be queried using standard aggregation pipelines.
Visualisations of these metrics (e.g., histograms, time-series plots and cohort charts) can be generated in external notebooks or dashboards to identify bottlenecks, outliers and opportunities for improvement.

Importantly, this log data is also the starting point for future machine-learning work.
For example, historical matches, completion outcomes and ratings can be used to train models that predict \emph{which} technician is likely to perform best for a given request, or to power an intelligent support chatbot that suggests next steps to users.

\section{Pilot Simulation}

Due to constraints on running a fully fledged field pilot within the project timeline, a simulated pilot was conducted:

\begin{itemize}
    \item A set of dummy consumer accounts (representing households in two villages) was created.
    \item A small pool of technicians with different skills, service radii and baseline ratings was configured.
    \item A sequence of service requests spanning different appliances and issue types was generated over a notional time period.
\end{itemize}

The matching algorithm and job flows were then exercised end-to-end.
During this simulation, status transitions were recorded in real time, and hypothetical ratings were assigned to explore how technician reputation evolves as more jobs are completed.

\section{Results and Observations}

The simulation highlights several encouraging patterns:

\begin{itemize}
    \item The scoring-based matching strategy strikes a reasonable balance between minimising travel distance and respecting skill and brand tags.
    \item Average response times remain within a few hours, even when technicians are allowed a grace window to accept or decline jobs.
    \item The admin dashboard views make it easy to spot outlier jobs (for example, those stuck in ``Created'' or ``Accepted'' for too long) and intervene manually.
\end{itemize}

At the same time, the pilot surfaces areas for improvement:

\begin{itemize}
    \item Notification reliability must be improved under intermittent connectivity, for example by adding queued SMS fallbacks to app or web notifications.
    \item Status labels should be simplified and localised for end users, with clear explanations of what each state means in practice.
    \item The matching algorithm could benefit from learning-based adjustments that incorporate success histories rather than relying only on hand-tuned weights.
\end{itemize}

\section{Qualitative Feedback}

Informal user testing with friends and family playing the roles of consumers and technicians yields useful qualitative feedback:

\begin{itemize}
    \item Consumers appreciate being able to see whether their request has been seen and accepted by a technician, but they would like status labels and key instructions in local languages.
    \item Technicians like having a consolidated view of upcoming jobs but request offline access to job details when travelling or in low-signal areas.
    \item Admin users find the job overview helpful but suggest additional filters (for example, by appliance category, severity or village cluster).
\end{itemize}

These insights feed directly into the future work roadmap, especially around improving usability, offline behaviour and the design of data-driven features.

%=====================================================
\chapter{Conclusion and Future Work}
\label{chap:conclusion}

\section{Summary of Contributions}

This project set out to explore whether a digital marketplace could meaningfully improve the last-mile delivery of appliance service in rural India.
Its main contributions are:

\begin{itemize}
    \item A structured understanding of the rural after-sales ecosystem, including stakeholder roles, current repair practices and pain points grounded in survey data.
    \item A system design for LastMile Service, including architecture, data model, matching and reputation mechanisms tailored to rural constraints.
    \item Implementation of a full-stack MVP that allows consumers to request service, technicians to manage jobs and admins to monitor operations.
    \item Basic analytics and a simulated pilot that demonstrate how operational metrics such as response and resolution times can be measured from platform logs.
    \item A preliminary business and partner ecosystem framing that situates LastMile within the broader landscape of brands, aggregators and device-care platforms.
\end{itemize}

\section{Limitations}

The work presented in this report has important limitations:

\begin{itemize}
    \item The pilot is simulated rather than fully deployed in a rural setting; real-world behaviour under connectivity, trust and cash-flow constraints remains to be validated.
    \item Offline-first capabilities are limited; the current MVP assumes at least intermittent connectivity for core flows such as job creation and updates.
    \item Integrations with brand CRM systems, warranty databases and spare-part ordering platforms are not yet implemented.
    \item Machine-learning components such as data-driven matching, demand forecasting or conversational assistance are only outlined conceptually and not yet trained or deployed.
\end{itemize}

\section{Future Work}

Promising directions for future work include:

\begin{itemize}
    \item Extending the MVP into a production-grade platform with native Android support, robust offline handling and stronger notification guarantees.
    \item Collaborating with a small set of brands, NGOs or local entrepreneurs to run a live district-level pilot and measure real-world impact on repair times, costs and technician livelihoods.
    \item Training data-driven models on historical logs to improve technician matching (learning the weights in the scoring function), estimate no-show risk and forecast demand by appliance category and locality.
    \item Developing an intelligent support chatbot that can operate in multiple Indian languages, guide users through request creation, perform basic troubleshooting and answer questions about pricing, safety and warranty.
    \item Building a structured learning layer for technicians, including micro-courses, quizzes and digital badges, and linking this progression to brand-recognised certification flows.
    \item Exploring policy and governance questions around data ownership, technician rights and consumer protection when rural repair activity becomes more digitally mediated.
\end{itemize}

\section*{Closing Remarks}

LastMile Service demonstrates that with thoughtful design, it is possible to bring the structure and transparency of digital platforms into the messy realities of rural appliance repair.
While the journey from academic prototype to deployed solution is long, this project provides a concrete starting point, a technical foundation and a vocabulary for further collaboration between engineers, brands, technicians and rural communities.
The data and architecture put in place also open the door to richer machine-learning and chatbot-based features that can gradually make the platform smarter, more personalised and more accessible over time.

%=====================================================
\bibliographystyle{plain}
\nocite{*} 
\bibliography{references} % Create references.bib with your sources

\end{document}
